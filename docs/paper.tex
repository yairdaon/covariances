
\documentclass[paper=a4, fontsize=11pt]{scrartcl} % A4 paper and 11pt font size

\usepackage[T1]{fontenc} % Use 8-bit encoding that has 256 glyphs
\usepackage[english]{babel} % English language/hyphenation
\usepackage{amsmath,amsfonts,amsthm} % Math packages
\usepackage{cite}
\usepackage{graphicx}
\usepackage{algorithm} % algorithm package
\usepackage[noend]{algpseudocode}
\usepackage{hyperref}

\usepackage{sectsty} % Allows customizing section commands
\allsectionsfont{\centering \normalfont\scshape} % Make all sections centered, the default font and small caps

\usepackage{fancyhdr} % Custom headers and footers
\pagestyle{fancyplain} % Makes all pages in the document conform to the custom headers and footers
\fancyhead{} % No page header - if you want one, create it in the same way as the footers below
\fancyfoot[L]{} % Empty left footer
\fancyfoot[C]{} % Empty center footer
\fancyfoot[R]{\thepage} % Page numbering for right footer
\renewcommand{\headrulewidth}{0pt} % Remove header underlines
\renewcommand{\footrulewidth}{0pt} % Remove footer underlines
\setlength{\headheight}{13.6pt} % Customize the height of the header

%%%%%%%%%%%%%%%%%%%%%%%%%%%%%%%%%%%%%%%%%%%%%
\DeclareMathOperator*{\argmin}{arg\,min}
\newtheorem{theorem}{Theorem}[section]
\newtheorem{lemma}[theorem]{Lemma}
\newtheorem{proposition}[theorem]{Proposition}
\newtheorem{corollary}[theorem]{Corollary}
%%%%%%%%%%%%%%%%%%%%%%%%%%%%%%%%
\numberwithin{equation}{section} % Number equations within sections (i.e. 1.1, 1.2, 2.1, 2.2 instead of 1, 2, 3, 4)
\numberwithin{figure}{section} % Number figures within sections (i.e. 1.1, 1.2, 2.1, 2.2 instead of 1, 2, 3, 4)
\numberwithin{table}{section} % Number tables within sections (i.e. 1.1, 1.2, 2.1, 2.2 instead of 1, 2, 3, 4)

\setlength\parindent{0pt} % Removes all indentation from paragraphs - comment this line for an assignment with lots of text





%----------------------------------------------------------------------------------------
% new commands
%----------------------------------------------------------------------------------------
\newcommand{\der}{\text{d}}
\newcommand{\coder}[1]{\texttt{#1}}
\newcommand{\inner}[2]{#1 \cdot #2}
\newcommand{\Op}{\mathcal{L}}
\newcommand{\var}{\text{Var}}
\newcommand{\corr}{\text{Corr}}
\newcommand{\dom}{\mathcal{D}}
\newcommand{\cov}{\text{Cov}}
\newcommand{\diag}{\text{diag}}
\newcommand{\Dom}{\mathcal{D}}
\title{A note on boundary conditions for covariance matrices derived from PDE operators}

\author{Yair Daon, Georg Stadler}
% \address[cims]{Courant Institute of Mathematical Sciences, New
%   York University, New York, NY, USA}

\date{}

\pdfinfo{%
  /Title    ()
  /Author   (Yair Daon, Georg Stadler)
  /Creator  ()
  /Producer ()
  /Subject  ()
  /Keywords ()
}

\begin{document}
\maketitle
\begin{abstract}
We present a method to remove boundary effects when using a differential operator as a precision
operator in bayesian inverse problems.
\end{abstract}

\section{Introduction}
The Bayesian approach to inifinite dimensional inverse problems 
requires the use of priors over function spaces. 
The theory underlying the choice of priors was laid in \cite{stuart2010inverse}. There,
it is suggested to use (fractional) powers of ``Laplacian-like'' operators as 
precision operators for Gaussian measures over function spaces. 
This gives rise to the interpretation of
the corresponding Green's function as the the covaraince funciton of the Gaussian prior.
It was noted in \cite{bui2013computational} that this choice of precision operator
gives rise to undesirable artifacts near the boundary. In this work we aim to 
remove these boundary effects. This means we want the above-mentioned Green's function
to be as similar as possible to the free space fundamental solution of the 
precision operator.

\section{Preliminaries}

Throught this paper, the term fundamental solution is reserved for the free space Green's 
function of a differential operator. The term Green's function 
is reserved exclusively for the one arising from a differential
operator on a domain $\Omega$ with prescribed boundary conditions.
Let $\Omega \subseteq \mathbb{R}^d$. Define a precision operator
by $\Op := (-\Delta + \kappa^2 )^{-p}$, where $\kappa^2 > 0$
and $p > \frac{d}{2}$ \cite{stuart2010inverse}. Of course, we
need to equip this operator with a domain (equivalently - boundary conditions). If we
take $\Dom(\Op) = \{ u \in H^2 | u \equiv 0 \text{ on } \partial \Omega \}$ we
get Homogeneous Dirichlet boundary. This has implicaltions for the Green's function
$G(x,y) = \Op^{-1}\delta_{y}(x)$. If $x,y$ are close to the boundary, then $G(x,y)$
is much smaller compared to other points with the same distance that are not close 
to the boundary. This happens because of the Dirichlet boundary condition - $G(x,y)$ has
to be zero on the boundary. If we take
 $\Dom(\Op) = \{ u \in H^2 | \frac{\partial u}{\partial n} \equiv 0 \text{ on } \partial \Omega \}$,
the opposite occurs. The Neumann boundary condition forces $G(x,y)$ to be almost
constant near the boundary. This was observed in \cite{bui2013computational}. Our goal
in this note is to find a boundary condition that makes the Green's function and the
fundamental solution of $\Op$ as similar as possible.

\subsection{Matern as a Fundamental Solution}
It is known \cite{lindgren2011explicit} that the fundamental solution in $\mathbb{R}^d$ of

$$
(-\Delta + \kappa^2)^{ \nu + d/2 }
$$

is the Matern covariance function:

$$
m(x,y; \kappa, \nu ) = \frac{\sigma^2}{2^{\nu-1}\Gamma(\nu)} (\kappa||x-y||)^{\nu} K_{\nu}( \kappa||x-y||), 
$$

with $\nu, \kappa > 0$ and

$$
\sigma^2 = \frac{\Gamma(\nu)}{\Gamma(\nu + d/2) (4\pi)^{d/2} \kappa^{2\nu}}.
$$

The following identity

\begin{align}\label{diff}
  \frac{\text{d}}{\text{d}x} (x^{\nu} K_{\nu}(x)) &= -x^{\nu}K_{\nu-1}(x), \ \forall \nu \in \mathbb{Z},
\end{align}

facilitates calculation of the gradient whenever $\nu \in \mathbb{Z}$:
\begin{align}\label{grad_matern}
  \begin{split}
  \nabla_{x} m( x,y; \kappa, \nu ) &= \nabla_{x} \frac{\sigma^2}{2^{\nu-1}\Gamma(\nu)} (\kappa||x-y||)^{\nu}K_{\nu}(\kappa||x-y||) \\
  %
  %
  %
  &= -\frac{\sigma^2}{2^{\nu-1}\Gamma(\nu)} (\kappa||x-y||)^{\nu}K_{\nu-1}(\kappa||x-y||) \nabla_{x} \kappa||x-y|| \\
  % 
  %
  %
  &= -\kappa\frac{\sigma^2}{2^{\nu-1}\Gamma(\nu)} (\kappa||x-y||)^{\nu}K_{\nu-1}(\kappa||x-y||) \frac{x-y}{||x-y||} \\
  \end{split}
\end{align}

This will be used later when we make reference to the fundamental solution of the operator.

\section{Exact Solution in $d=1$ using Robin BC}
The problem we are dealing with can be solved exactly in 1D, if one chooses the right
Robin boundary condition.
Recall that the fundamental solution for $\Op = -\Delta + \kappa^{2}$ on $\mathbb{R}$ is $\Phi(x,y) = \frac{\exp( -\kappa |x-y| ) }{2\kappa}$. Note that since $p =1 > \frac{d}{2}$, this is a valid precision operator \cite{stuart2010inverse}. Fix $x \in \Omega := (0,1)$ and define $H:=  \Phi - G$,  \cite{evans} calls this a corrector function. 
Consider a homogeneous Robin boundary condition
for $G$ on $\partial \Omega$ so that $\beta G + \frac{ \partial G}{\partial n_{y}} = 0$.
Clearly $H$ satisfies:

\begin{align}\label{difference equation}
  \begin{split} 
    (-\Delta_{y} + \kappa^2)H(x,y) &= 0, \forall y\in \Omega \\
    %
    %
    %     
    \beta H(x,y) + \frac{\partial H}{\partial n_{y}}(x,y) &= \beta \Phi(x,y) + \frac{\partial \Phi}{\partial n_{y}}(x,y), \forall y \in \partial \Omega.
  \end{split}
\end{align}
On $\partial \Omega = \{0,1\}$ we observe that

\begin{align*}
  \begin{split}
    \beta \Phi(0,x) + \frac{\partial \Phi}{\partial n_{y}}|_{y=0}(0,x) &= \frac{1}{2}\exp( -\kappa x)[ \frac{\beta}{\kappa} - 1] \\
    %
    %
    %
    \beta \Phi(1,x) + \frac{\partial \Phi}{\partial n_{y}}|_{y=1}(1,x) &= \frac{1}{2} \exp (-\kappa (1-x)) [\frac{\beta}{\kappa} - 1 ].\\
  \end{split}
\end{align*}

Setting $\beta = \kappa$, conclude that $H$ satisfies

\begin{align*}
  \begin{split}
    (-\Delta_{y} + \kappa^2 ) H(x,y) &= 0 , y\in \Omega \\
    % 
    % 
    % 
    \kappa H(x,y) + \frac{\partial H}{\partial n_{y}}(x,y) &= 0 , y\in \partial \Omega.\\
  \end{split}
\end{align*}

Thus, $\Phi - G = H \equiv 0$ and we have a perfect solution.

\section{Approximate Solution in $d >1$}

In higher dimensions, we don't expect to have a perfect matching so we turn to approximations.
Again, we denote $\Op$ the precision operator we use, only with homogeneous Robin boundary condition.
In this section, we may take 
$\Op_{1} = (-\Delta + \kappa^2)$ or $\Op_{2} = (-\Delta + \kappa^2)^2$, 
according to the dimension of our domain. If $d=2$, we may use either $\Op_{1}$ or $\Op_{2}$ (even though the
former does not make sense as a precision operator). When $d=3$ the integrals in the procedure below are infinite
(see subsection \ref{integrability} below), so we
may only use $\Op_{2}$ (and higher powers). Denote $\Phi$ the fundamental
solution for $\Op$. Fix $x\in \partial \Omega$. Denote, again $H = \Phi - G$. Equation \ref{difference equation}
still holds. If we can make $\beta H + \frac{ \partial H}{\partial n_{x}} =0$ then $H\equiv 0$ and
a perfect solution is found. Since we can't do that, we seek to minimize this rhs in \ref{difference equation},
for all $y\in \Omega$. Thus, we define the optimal $\beta$ as follows:

\begin{align}
  \begin{split}
    \beta^{*}(x) :&= \argmin_{\beta > 0} \frac{1}{2} \int_{\Omega} (\beta \Phi(x,y) + \frac{\partial \Phi}{\partial n_{x}} )^{2} dy \\
    % 
    % 
    % 
    &= \argmin_{\beta > 0} \frac{\beta^2}{2} \int_{\Omega} \Phi^2(x,y) dy+\beta \int_{\Omega} \Phi(x,y) \frac{\partial \Phi}{\partial n_{x}} dy. \\
  \end{split}
\end{align}

The idea is simple - instead of making the right hand side of the boundary condition zero, we seek to make it as small as possible for 
every $x\in \partial \Omega$. This quadratic has a minimum since its leading coefficient 
is positive. The minimum is found to be:

$$
\beta^{*}(x) = - \frac{\int_{\Omega}  \Phi(x,y) \frac{\partial \Phi}{\partial n_{x}} dy}{\int_{\Omega}  \Phi^2(x,y) dy}.
$$

% Recall that for a fixed $x$:

% $$
% \nabla_x\Phi(x,y) = -\frac{\kappa}{2\pi} K_1( \kappa ||x-y|| ) \frac{x-y}{||x-y||}.
% $$

\subsection{Integrability}\label{integrability}
Here we show that we can use the above formula for $\Op{1}$ when $d=2$ but not when $d=3$.
$\Op_{2}$ is OK for both, since it's green's function has no singularity in neither $d=2$ nor $d=3$.
Take WLOG $0=x\in \partial \Omega$ and consider half of the ball $B(0,\epsilon)$.
Since integtrability doesn't change if we take half of the ball or all of it, we just integrate in $y\in B(0,\epsilon)$. We take the boundary
of our domain s.t.  $\hat{n}_x = (-1,0,0)^t$.  If we take $\epsilon$ small enough, we can ignore $\kappa$, which will make the calculations
easier. Below, $C$ denotes a positive constant we don't care about. The constant can change between lines. 
Start with 3D problem and recall that in spherical coordinates $dy = r^2 \sin \theta dr d\theta d \phi$:

\begin{align*}
    \int_{B(0,\epsilon)} |m^2(y,0)|dy &= C\int_{B(0,\epsilon)} \frac{1}{||y||^2} dy \\
    &= C \int_{0}^{\epsilon} \frac{r^2}{r^2} dr \\
    &= C < \infty \\
    \int_{B(0,\epsilon)} |\Phi(0,y) \frac{\partial \Phi(0,y)}{\partial n_x}| dy &= C\int_{B(0,\epsilon)} |\frac{1}{||y||} (-\frac{1}{||y||^2} \frac{-y\cdot (-1,0,0)^t}{||y||})| dy\\
    &= C \int_{0}^{\epsilon} \frac{r }{r^4}r^2  dr  \\
    &= C \int_{0}^{\epsilon} \frac{1}{r} dr \\
    &= C \log r|_{r=0}^{r=\epsilon} \\
    & = \infty,
\end{align*}

Now the 2D problem. Recall that in polar coordinates $dy = r\sin \theta dr d\theta$:

\begin{align*}
     \int_{B(0,\epsilon)} |m^2(y,0)|dy &= C \int_{B(0,\epsilon)} \log^2||y|| dy \\
    &= C \int_{0}^{\epsilon} r \log^2 r dr \\
    &= C < \infty, \\
    \int_{B(0,\epsilon)} |\Phi(0,y) \frac{\partial \Phi(0,y)}{\partial n_x}| dy &= C \int_{B(0,\epsilon)} |\log||y|| (\frac{1}{||y||} \frac{-y\cdot (-1,0)^t}{||y||})| dy\\
    &= C\int_{0}^{\epsilon} |\log r\frac{r }{r^2}r| dr\\
    &= C \int_{0}^{\epsilon} |\log r| dr \\
    &= -C(r \log |r| - r)|_{0}^{\infty} < \infty.
\end{align*}

The above problem is uninformative in 3 dimensions but has a solution in 2 dimensions.

\section{Results}
In this section we present results found using our method. We use FENiCS \cite{logg2012automated} for
the visualizations presented below. Everything required to make these may be found on 
\href{https://github.com/yairdaon/covariances}{GitHub}.

\bibliographystyle{unsrt}
\bibliography{refs.bib}
\end{document}


\subsection{Variational Formulation}
When using Finite Element, one needs to derive a variational formulation for the problem they are using. It is important 
to check that this variational formulation makes sense from an optimization perspective. This is the 
goal of this section. We show how a Robin problem arises from a minimization problem.
Take $I[u] = \frac{1}{2}\int_{\Omega} ||\nabla u||^{2} + \frac{1}{2}\kappa^{2}u^2 - fu + \frac{1}{2}\int_{\partial \Omega} \beta( u-\frac{g}{\beta})^2, \beta > 0$ and define
$i(\tau) = I[u + \tau v]$, with $u,v \in H^{1}(\Omega)$. Note that if $\beta < 0$ then $I$ is not bounded below, so the minimization
problem we consider shortly makes no sense.

\begin{align*}
  i(\tau) &= I[u + \tau v ] \\
  %
  %
  % 
  &= \frac{1}{2}\int_{\Omega} ||\nabla (u+ \tau v)||^{2} + \frac{1}{2}\kappa^{2}(u+ \tau v)^2 - f(u + \tau v) + \frac{1}{2}\int_{\partial \Omega} \beta(u- \frac{g}{\beta}+ \tau v)^2 \\
  %
  %
  %
  &= I[u] + \tau ( \int_{\Omega} \nabla u \cdot \nabla v + \kappa^{2}uv - fv + \int_{\partial \Omega} \beta( u - \frac{g}{\beta}) v ) +\mathbf{o} (\tau)\\
  % 
  %
  %
  &= I[u] + \tau ( \int_{\Omega} \nabla u \cdot \nabla v + \kappa^{2}uv - fv + \int_{\partial \Omega} (\beta u - g) v ) +\mathbf{o} (\tau).\\ 
\end{align*}

If $u$ is indeed a minimum for $I$ , then $i'(0) = 0$. The corresponding variational problem is:

\begin{align*}
\int_{\Omega} \nabla u \cdot \nabla v + \kappa^{2}uv + \int_{\partial \Omega} \beta u v  =\int_{\Omega} fv +\int_{\partial \Omega} gv. 
\end{align*}

Formally integrating the first term by parts gives:
$$
\int_{\Omega}  (-\Delta u + \kappa^{2} u) v + \int_{\partial \Omega} ( \beta u + \frac{ \partial u }{\partial n } )v = \int_{\Omega} fv + \int_{\partial \Omega} gv.
$$
We see the Robin condition arises naturally from the minimization of $I$. 
