
\documentclass[paper=a4, fontsize=11pt]{scrartcl} % A4 paper and 11pt font size

\usepackage[T1]{fontenc} % Use 8-bit encoding that has 256 glyphs
\usepackage[english]{babel} % English language/hyphenation
\usepackage{amsmath,amsfonts,amsthm} % Math packages
\usepackage{cite}
\usepackage{graphicx}
\usepackage{algorithm} % algorithm package
\usepackage[noend]{algpseudocode}

\usepackage{sectsty} % Allows customizing section commands
\allsectionsfont{\centering \normalfont\scshape} % Make all sections centered, the default font and small caps

\usepackage{fancyhdr} % Custom headers and footers
\pagestyle{fancyplain} % Makes all pages in the document conform to the custom headers and footers
\fancyhead{} % No page header - if you want one, create it in the same way as the footers below
\fancyfoot[L]{} % Empty left footer
\fancyfoot[C]{} % Empty center footer
\fancyfoot[R]{\thepage} % Page numbering for right footer
\renewcommand{\headrulewidth}{0pt} % Remove header underlines
\renewcommand{\footrulewidth}{0pt} % Remove footer underlines
\setlength{\headheight}{13.6pt} % Customize the height of the header

%%%%%%%%%%%%%%%%%%%%%%%%%%%%%%%%%%%%%%%%%%%%%
\DeclareMathOperator*{\argmin}{arg\,min}
\newtheorem{theorem}{Theorem}[section]
\newtheorem{lemma}[theorem]{Lemma}
\newtheorem{proposition}[theorem]{Proposition}
\newtheorem{corollary}[theorem]{Corollary}
%%%%%%%%%%%%%%%%%%%%%%%%%%%%%%%%
\numberwithin{equation}{section} % Number equations within sections (i.e. 1.1, 1.2, 2.1, 2.2 instead of 1, 2, 3, 4)
\numberwithin{figure}{section} % Number figures within sections (i.e. 1.1, 1.2, 2.1, 2.2 instead of 1, 2, 3, 4)
\numberwithin{table}{section} % Number tables within sections (i.e. 1.1, 1.2, 2.1, 2.2 instead of 1, 2, 3, 4)

\setlength\parindent{0pt} % Removes all indentation from paragraphs - comment this line for an assignment with lots of text





%----------------------------------------------------------------------------------------
% new commands
%----------------------------------------------------------------------------------------
\newcommand{\der}{\text{d}}
\newcommand{\coder}[1]{\texttt{#1}}
\newcommand{\inner}[2]{#1 \cdot #2}
\newcommand{\Op}{\mathcal{L}}

\title{Gaussian measures with an operator that ignores the boundary as covariance}

\author{Yair Daon}
\date{}

\pdfinfo{%
  /Title    ()
  /Author   (Yair Daon)
  /Creator  ()
  /Producer ()
  /Subject  ()
  /Keywords ()
}

\begin{document}
\maketitle
\begin{abstract}
\end{abstract}

\section{Problem statement}
Say one has a domain $\Omega \subseteq \mathbb{R}^d$. One would like to consider a covariance operator on the domain 
that does not ``see'' the boundary. This means that there are no strong correlations / decorrelations close to
the boundary. The covariances that are considered in this context are ``laplacian-like operators''. Their properties
are studied by \cite{stuart2010inverse}.
We try to take $(-\Delta + \kappa^2 )^{-p}$, where $\kappa^2 > 0$
and $p$ staisfies the restriction stated by \cite{stuart2010inverse}. Usually $p > \frac{d}{2}$,
where $d$ is the dimensionality defined above. If one takes Dirichlet boundary conditions, one obeserves 
strong decorrelations for points near the boundary. For Neumann boundary, the opposite occurs. We try 
to remove or ameliorate these boundary effects \cite{bui2013computational}.

Below, a Robin BC approach is considered. We seek a boundary condition of the Robin family that performs well (in some sense to be defined).
The term fundamental solution is reserved for the free space Green's function of a differential operator. The term Green's function 
is reserved exclusively for the one arising from a differential operator on a domain with prescribed boundary conditions.

\section{General Info}

\subsection{Fundamental Solution and Green's Funciton}
Fix $\kappa,d, y\in \Omega$ and $\beta > 0$. Denote $G$ the Green's function. 
\begin{align}
  \begin{split}
    &(-\Delta_{x} + \kappa^2 ) G(x,y) = \delta_{y}(x),  x \in  \Omega, \\
    &\beta G(x,y) + \frac{\partial G}{\partial n_{x}}(x,y) = 0, x \in \partial \Omega, 
  \end{split}
\end{align}

which can be shown to be symmetric. With the same notation, define the fundamental solution

\begin{align}
  \begin{split}
    (-\Delta_{x} + \kappa^2) \Phi(x,y) &= \delta_{y}(x), x \in \mathbb{R}^d. \\
  \end{split}
\end{align}
It will be clear from the context which operator is being used.
% \subsection{Known Results for $d = 1,2,3$}

% It is known that:
% \begin{itemize}
% \item If $d = 1$, then $\Phi(x,y) = \frac{ \exp( -\kappa||x-y|| )}{2\kappa}$. To see this, take WLOG $y=0$ and $\Phi(x) = \frac{\exp(-\kappa |x| )}{2\kappa}$. 
%   \begin{align}
%     \begin{split}
%       \kappa \frac{d^{2}m}{dx^2} &= \frac{d}{dx} [-\kappa H(x) \exp( -\kappa |x|)] \\
%       %
%       %
%       %
%       & = (-\kappa H(x))^2 \exp( -\kappa |x| ) -\kappa \exp(-\kappa |x| ) \delta (x) \\ 
%       % 
%       %
%       %
%       & = \kappa^2 \exp( -\kappa |x| ) -\kappa \exp(-\kappa |x| ) \delta (x) \\   
%     \end{split}
%   \end{align}
%   and it is easy to see now that $(-\Delta + \kappa^2) \Phi(x) = \delta (x)$. 
% \item If $d = 2$, then $\Phi(x,y) = \frac{1}{2\pi} K_{0}( \kappa ||x-y|| ) $ \cite{wikifundamental}. By
%   \cite[page 79, number (7)]{watson1995treatise}, $\frac{dK_{0}(x)}{dx} = -K_{1}(x) = -K_{-1}(x)$. Also
%   \begin{align*}
%     \nabla_{x} \Phi(x,y) &= \nabla_{x} \frac{1}{2\pi} K_{0}( \kappa ||x-y|| )\\
%     &= -\frac{\kappa}{2\pi} K_{1}( \kappa ||x-y|| ) \nabla_{x}||x-y|| \\
%     &= -\frac{\kappa}{2\pi} K_{1}(\kappa ||x-y|| ) \frac{x-y}{||x-y||}
%   \end{align*}
%   \item If $d = 3$, then $\Phi(x,y) = \frac{ \exp( -\kappa||x-y|| )}{4\pi||x-y||}$ \cite{wikifundamental}.
% \end{itemize}

\subsection{Matern as a Fundamental Solution}
It is known \cite{lindgren2011explicit} that the fundamental solution in $\mathbb{R}^d$ of

$$
(-\Delta + \kappa^2)^{ \nu + d/2 }
$$

is:

$$
m(x,y; \kappa, \nu ) = \frac{\sigma^2}{2^{\nu-1}\Gamma(\nu)} (\kappa||x-y||)^{\nu} K_{\nu}( \kappa||x-y||), 
$$

with $\nu, \kappa > 0$ and

$$
\sigma^2 = \frac{\Gamma(\nu)}{\Gamma(\nu + d/2) (4\pi)^{d/2} \kappa^{2\nu}}.
$$

Note that by the identity below, if $\nu \in \mathbb{Z}$

\begin{align}\label{grad_matern}
  \begin{split}
  \nabla_{y} m( x,y; \kappa, \nu ) &= \nabla_{y} \frac{\sigma^2}{2^{\nu-1}\Gamma(\nu)} (\kappa||x-y||)^{\nu}K_{\nu}(\kappa||x-y||) \\
  %
  %
  %
  &= -\frac{\sigma^2}{2^{\nu-1}\Gamma(\nu)} (\kappa||x-y||)^{\nu}K_{\nu-1}(\kappa||x-y||) \nabla_{y} \kappa||x-y|| \\
  % 
  %
  %
  &= \kappa\frac{\sigma^2}{2^{\nu-1}\Gamma(\nu)} (\kappa||x-y||)^{\nu}K_{\nu-1}(\kappa||x-y||) \frac{x-y}{||x-y||} \\
  \end{split}
\end{align}

\subsection{Identities}

\begin{align}\label{diff}
  \frac{\text{d}}{\text{d}x} (x^{\nu} K_{\nu}(x)) &= -x^{\nu}K_{\nu-1}(x), \ \forall \nu \in \mathbb{Z}
\end{align}




\begin{align*}
  \begin{split}
    \int_{\Omega} u( \kappa^2 - \Delta) v - v( \kappa^2 - \Delta) u &= \int_{\Omega} v\Delta u - u\Delta v\\
    % 
    % 
    % 
    &= \int_{\partial \Omega } v\frac{\partial u}{\partial n} - u \frac{\partial v}{\partial n}\\
\end{split}
\end{align*}

% \subsection{Representation Formula}
% We want to find the solution to
% \begin{align*}
%   &(-\Delta + \kappa^2 ) u = f \text{ in } \Omega, \\
%   &\beta u + \frac{\partial u}{\partial n}=0  \text{ in } \partial \Omega. 
% \end{align*}

% Fix $y\in \Omega$. From the (modified) Green's identity above:

% \begin{align*}
%     u(y) &= \int_{\Omega} u(x) \delta_y(x) dx \\
%     % 
%     % 
%     % 
%     &= \int_{\Omega}  u(x)(-\Delta_{x} +\kappa^2 )G(x,y) dx \\
%     % 
%     % 
%     % 
%     & = \int_{\Omega} G(x,y) f(x) dx + \int_{\partial\Omega} G(x,y) \frac{\partial u}{\partial n_{x}} - u(x) \frac{\partial G(x,y)}{\partial n_{x}} dS(x) \\
%     % 
%     % 
%     % 
%     & = \int_{\Omega} G(x,y) f(x) dx + \int_{\partial\Omega}   G(x,y) (-\beta u(x)) - u(x) \frac{\partial G(x,y)}{\partial n_{x}} dS(x) \\
%     % 
%     % 
%     % 
%     & = \int_{\Omega} G(x,y) f(x) dx - \int_{\partial\Omega} u(x) [ \beta  G(x,y)  + \frac{\partial G(x,y)}{\partial n_{x}} ] dS(x) \\
%     %
%     %
%     % 
%     & = \int_{\Omega} G(x,y) f(x) dx,
% \end{align*}

% which is the required representation formula. The third equality is sometimes called the Kirchhoff- Helmholtz representation.
% This is indeed the solution we saught: 
% \begin{align}
%   \begin{split}
%     %
%     % 
%     % 
%     (-\Delta_{y} + \kappa^2 ) u(y) &= \int_{\Omega} (-\Delta_{y} +\kappa^2 )G(x,y)f(x)dx \\
%     %
%     % 
%     % 
%     & = \int_{\Omega} \delta_{x}(y) f(x)dx \\
%     %
%     % 
%     % 
%     & = \int_{\Omega} \delta_{0}(x-y) f(x)dx \\ 
%     % 
%     % 
%     % 
%     & = f(y) \text{ for } y \in \Omega\\
%     %
%     % 
%     % 
%     \beta u(y) & =\int_{\Omega}  \beta G(x,y)f(x)dx \text{ (now } y\in \partial \Omega).\\
%     %
%     % 
%     % 
%     & =- \int_{\Omega} \frac{\partial G}{\partial n_{y}}f(x)dx \\
%     %
%     % 
%     % 
%     & =- \frac{\partial}{\partial n_{y}}  \int_{\Omega} G(x,y)f(x)dx \\
%     %
%     % 
%     % 
%     & =- \frac{\partial u} {\partial n}(y) 
%   \end{split}
% \end{align}


\subsection{Variational Formulation}
Take $I[u] = \frac{1}{2}\int_{\Omega} ||\nabla u||^{2} + \frac{1}{2}\kappa^{2}u^2 - fu + \frac{1}{2}\int_{\partial \Omega} \beta( u-\frac{g}{\beta})^2, \beta > 0$ and define
$i(\tau) = I[u + \tau v]$, with $u,v \in H^{1}(\Omega)$. Note that if $\beta < 0$ then $I$ is not bounded below, so the minimization
problem we consider shortly makes no sense.

\begin{align*}
  i(\tau) &= I[u + \tau v ] \\
  %
  %
  % 
  &= \frac{1}{2}\int_{\Omega} ||\nabla (u+ \tau v)||^{2} + \frac{1}{2}\kappa^{2}(u+ \tau v)^2 - f(u + \tau v) + \frac{1}{2}\int_{\partial \Omega} \beta(u- \frac{g}{\beta}+ \tau v)^2 \\
  %
  %
  %
  &= I[u] + \tau ( \int_{\Omega} \nabla u \cdot \nabla v + \kappa^{2}uv - fv + \int_{\partial \Omega} \beta( u - \frac{g}{\beta}) v ) +\mathbf{o} (\tau)\\
  % 
  %
  %
  &= I[u] + \tau ( \int_{\Omega} \nabla u \cdot \nabla v + \kappa^{2}uv - fv + \int_{\partial \Omega} (\beta u - g) v ) +\mathbf{o} (\tau).\\ 
\end{align*}

If $u$ is indeed a minimum for $I$ , then $i'(0) = 0$. The corresponding variational problem is:

\begin{align*}
\int_{\Omega} \nabla u \cdot \nabla v + \kappa^{2}uv + \int_{\partial \Omega} \beta u v  =\int_{\Omega} fv +\int_{\partial \Omega} gv. 
\end{align*}

Formally integrating the first term by parts gives:
$$
\int_{\Omega}  (-\Delta u + \kappa^{2} u) v + \int_{\partial \Omega} ( \beta u + \frac{ \partial u }{\partial n } )v = \int_{\Omega} fv + \int_{\partial \Omega} gv.
$$
We see the Robin condition arises naturally from the minimization of $I$. 


% \section{1D Solution}
% Recall that the fundamental solution for $-\Delta + \kappa^{2}$ on $\mathbb{R}$ is $\Phi(x,y) = \frac{\exp( -\kappa |x-y| ) }{2\kappa}$.
% Fix $y \in \Omega := (0,1)$ and define $\bar{m}:=  m - G$,  Evans calls this a corrector function. 
% Consider a homogeneous Robin boundary condition
% for $G$ on $\partial \Omega$ so that $\beta G + \frac{ \partial G}{\partial n_{x}} = 0$.
% Once we plug in $\bar{m} = m - G$. we see that $\bar{m}$ satisfies:

% \begin{align}
%   \begin{split} 
%     (-\Delta_{x} + \kappa^2)\bar{m}(x,y) &= 0, \forall x\in \Omega \\
%     %
%     %
%     %     
%     \beta\bar{m}(x,y) + \frac{\partial \bar{m}}{\partial n_{x}}(x,y) &= \beta \Phi(x,y) + \frac{\partial m}{\partial n_{x}}(x,y), \forall x \in \partial \Omega.
%   \end{split}
% \end{align}
% On $\partial \Omega = \{0,1\}$ we observe that

% \begin{align}
%   \begin{split}
%     \beta \Phi(0,y) + \frac{\partial m}{\partial n_{x}}|_{x=0}(0,y) &= \frac{1}{2}\exp( -\kappa y)[ \frac{\beta}{\kappa} - 1] \\
%     %
%     %
%     %
%     \beta \Phi(1,y) + \frac{\partial m}{\partial n_{x}}|_{x=1}(1,y) &= \frac{1}{2} \exp (-\kappa (1-y)) [\frac{\beta}{\kappa} - 1 ].\\
%   \end{split}
% \end{align}

% Setting $\beta = \kappa$, conclude that $\bar{m}$ satisfies

% \begin{align}
%   \begin{split}
%     (-\Delta_{x} + \kappa^2 ) \bar{m}(x,y) &= 0 , x\in \Omega \\
%     % 
%     % 
%     % 
%     \kappa \bar{m}(x,y) + \frac{\partial \bar{m}}{\partial n_{x}}(x,y) &= 0 , x\in \partial \Omega.\\
%   \end{split}
% \end{align}

% Thus, $m - m_{\kappa} = \bar{m} \equiv 0$ and we have a perfect solution.

\section{Optimization Problem}

% \subsection{Formulation and motivation}
% In higher dimensions, we don't expect to have a perfect matching so we turn to approximations. For $x\in \partial \Omega$,
% one such possible approximation may be found by solving

% \begin{align}
%   \begin{split}
%     \beta^{*}(x) :&= \argmin_{\beta > 0} \frac{1}{2} \int_{\Omega} (\beta \Phi(x,y) + \frac{\partial m}{\partial n_{x}} )^{2} dy \\
%     % 
%     % 
%     % 
%     &= \argmin_{\beta > 0} \frac{\beta^2}{2} \int_{\Omega} m^2(x,y) dy+\beta \int_{\Omega} \Phi(x,y) \frac{\partial m}{\partial n_{x}} dy. \\
%   \end{split}
% \end{align}

% The idea is simple - instead of making the right hand side of the boundary condition zero, we seek to make it as small as possible for 
% every $x\in \partial \Omega$. This quadratic has a minimum $\hat{\beta}(x) \in \mathbb{R}$ since its leading coefficient 
% is positive. The minimum is found to be:

% $$
% \hat{\beta}(x) = - \frac{\int_{\Omega}  \Phi(x,y) \frac{\partial m}{\partial n_{x}} dy}{\int_{\Omega}  m^2(x,y) dy}.
% $$

% Recall that for a fixed $x$:

% $$
% \nabla_x\Phi(x,y) = -\frac{\kappa}{2\pi} K_1( \kappa ||x-y|| ) \frac{x-y}{||x-y||}.
% $$

% \subsection{Issue of Integrability}
% First, check the integrability of the above enumerator and denominator. Take WLOG $0=x\in \partial \Omega$ and consider half of the ball $B(0,\epsilon)$.
% Since integtrability doesn't change if we take half of the ball or all of it, we just integrate in $y\in B(0,\epsilon)$. We take the boundary
% of our domain s.t.  $\hat{n}_x = (-1,0,0)^t$.  If we take $\epsilon$ small enough, we can ignore $\kappa$, which will make the calculations
% easier. Below, $C$ denotes a positive constant we don't care about. The constant can change between lines. 
% Start with 3D problem and recall that in spherical coordinates $dy = r^2 \sin \theta dr d\theta d \phi$:

% \begin{align*}
%     \int_{B(0,\epsilon)} |m^2(y,0)|dy &= C\int_{B(0,\epsilon)} \frac{1}{||y||^2} dy \\
%     &= C \int_{0}^{\epsilon} \frac{r^2}{r^2} dr \\
%     &= C < \infty \\
%     \int_{B(0,\epsilon)} |\Phi(0,y) \frac{\partial \Phi(0,y)}{\partial n_x}| dy &= C\int_{B(0,\epsilon)} |\frac{1}{||y||} (-\frac{1}{||y||^2} \frac{-y\cdot (-1,0,0)^t}{||y||})| dy\\
%     &= C \int_{0}^{\epsilon} \frac{r }{r^4}r^2  dr  \\
%     &= C \int_{0}^{\epsilon} \frac{1}{r} dr \\
%     &= C \log r|_{r=0}^{r=\epsilon} \\
%     & = \infty,
% \end{align*}

% Now the 2D problem. Recall that in polar coordinates $dy = r\sin \theta dr d\theta$:

% \begin{align*}
%      \int_{B(0,\epsilon)} |m^2(y,0)|dy &= C \int_{B(0,\epsilon)} \log^2||y|| dy \\
%     &= C \int_{0}^{\epsilon} r \log^2 r dr \\
%     &= C < \infty, \\
%     \int_{B(0,\epsilon)} |\Phi(0,y) \frac{\partial \Phi(0,y)}{\partial n_x}| dy &= C \int_{B(0,\epsilon)} |\log||y|| (\frac{1}{||y||} \frac{-y\cdot (-1,0)^t}{||y||})| dy\\
%     &= C\int_{0}^{\epsilon} |\log r\frac{r }{r^2}r| dr\\
%     &= C \int_{0}^{\epsilon} |\log r| dr \\
%     &= -C(r \log |r| - r)|_{0}^{\infty} < \infty.
% \end{align*}

% The above problem is uninformative in 3 dimensions but has a solution in 2 dimensions.


%\subsection{An Optimization Problem}
Consider a differential operator $\Op$ with some BC. For $x\in \partial \Omega$, we'd like to find a BC 
such that $G(x,y) \approx \Phi(x,y)$. We may use an inhomogeneous Robin boundary condition,
$u + \beta \frac{\partial u}{\partial n} = g$ and take $\Op = (-\Delta + \kappa^2)^2$ with this Robin 
Boundary condition. Fix $x \in \partial \Omega$ and let $H^x(y) = G^{x}(y) - \Phi^{x}(y)$. Then $H^{x}$ satisfies

\begin{align*}
  &\Op H^{x} = 0 \text{ in } \Omega, \\
  &\beta H^{x} + \frac{\partial H^{x}}{\partial n} = g - \beta\Phi^{x}  - \frac{\partial \Phi^{x}}{\partial n}  \text{ in } \partial \Omega, 
\end{align*}

where $\beta, g$ and $\Phi^{x}$ are functions in $y$. For a fixed $y \in \partial \Omega$, we want $g(y) - \beta (y) \Phi^{x}(y) - \frac{\partial \Phi^{x}(y)}{\partial n} =0 \forall x \in \Omega$
Of course, we can't satisfy this, so we define
 $f(\beta, g) =\frac{1}{2} \int_{\Omega} ( \beta \Phi^{x}(y) + \frac{\partial \Phi^{x}(y)}{\partial n} - g)^{2} dx$.
and minimize.
\begin{align}
  \begin{split}
    \beta^{*}(y), g^{*}(y) :&= \argmin_{g \in \mathbb{R}, \beta > 0} f(\beta,g ) \\
    %
    %
    %
    &=\frac{1}{2} \int_{\Omega} ( \beta \Phi^{x}(y) + \frac{\partial \Phi^{x}(y)}{\partial n} - g)^{2} dx \\
    % 
    % 
    % 
    &= \argmin_{g, \beta } \frac{1}{2} \int_{\Omega} ( \beta \Phi(x,y) + \nabla_{y} \Phi(x,y) \cdot n_{y} - g)^{2} dx\\
    % 
    % 
    &= \argmin_{g, \beta } \frac{1}{2} \int_{\Omega} \beta^2 \Phi^2(x,y) + (\nabla_{y} \Phi(x,y) \cdot n_{y})^2 + g^2\\
    &\ \ \ - 2g\beta\Phi(x,y) + 2\beta \Phi(x,y)\nabla_{y} \Phi(x,y) \cdot n_{y} \\
    &\ \ \ - 2g\nabla_{y} \Phi(x,y) \cdot n_{y} dx \\
    % 
    % 
    % 
    &= \argmin_{g, \beta } \frac{1}{2} \int_{\Omega} \beta^2 \Phi^2(x,y) +  g^2dx\\
    &\ \ \ + \int_{\Omega} \beta \Phi(x,y)\nabla_{y} \Phi(x,y) \cdot n_{y} -g\nabla_{y} \Phi(x,y) \cdot n_{y}\\
    &\ \ \ -g\beta\Phi(x,y) dx \\
    % 
    % 
    % 
  \end{split}
\end{align}

To minimize we differentiate.

\begin{align*}
  \frac{\partial f}{\partial \beta} &= \int_{\Omega} \beta\Phi^2(x,y) + \Phi(x,y)\nabla_{y} \Phi(x,y) \cdot n_{y} -g\Phi(x,y) dx\\
  %
  %
  %
  \frac{\partial f}{\partial g} &= \int_{\Omega} g - \nabla_{y} \Phi(x,y) \cdot n_{y} -\beta\Phi(x,y) dx. \\
\end{align*}

Equating these partial derivatives to zero results in a $2\times 2$ set of equations.

\[
\begin{bmatrix}
  \int_{\Omega}\Phi^2(x,y) dx & -\int_{\Omega} \Phi(x,y) dx \\
  -\int_{\Omega}\Phi(x,y) dx    & \int_{\Omega}dx \\
\end{bmatrix}
\begin{bmatrix}
  \beta^{*}(y) \\
  g^{*}(y) \\
\end{bmatrix}
=
\begin{bmatrix}
  -\int_{\Omega} \Phi(x,y) \nabla_{y}\Phi(x,y) \cdot n_{y} dx \\
  \int_{\Omega} \nabla_{y}\Phi(x,y) \cdot n_{y}dx
\end{bmatrix}.
\]

Now, all we need to do is decide on an operator. This will give us the $\Phi$ as a matern covariance and we will also
be able to differentiate it. 

\subsection{Calculation in $\mathbb{R}^2$}

According to \cite{stuart2010inverse} we must take at least $(-\Delta + \kappa^2)^{\nu + d/2}$ with $\nu > 0$. Here I take 
$\Op =(-\Delta + \kappa^2 )^2$. In this case, $\nu = 1$ and so using \ref{grad_matern}

\begin{align*}
  \Phi(x,y) &= \sigma^2 \kappa||x-y|| K_{1}(\kappa ||x-y||) \\
  \nabla_{y}\Phi(x,y) &= \kappa  \sigma^2 \kappa||x-y|| K_{0}(\kappa ||x-y||) \frac{x-y}{||x-y||}.
\end{align*}


% \section{Optimization Problem}
% Naturally, one can seek $\beta$ that minimizes the difference between the Green's function and
% the fundamental solution:

% $$
% \beta^{*} := \argmin_{\beta > 0} \frac{1}{2} \int_{\Omega}\int_{\Omega} (\Phi(x,y) - G(x,y))^2 dx dy
% $$

% But this looks hard. We can use what we have.
% to do this we start from a different spot. Say one has a finite element discretization of the problem. Then one may want to minimize the difference between
% the operators. This can be done as follows. Consider the finite element basis functions $\{f_y\}_{y \in S }$. In the current context, the easiest
% way to think of $f_{y}$ is as an approximation to $\delta_y$ (a dirac delta centered at $y\in S$). Note that to have a complete analogy, we must 
% have $\int_{\Omega}f_y = 1, \forall y \in S$.
% Now define:

% $$
% (-\Delta + \kappa^2)u_y = f_y \text{ in } \mathbb{R}^d
% $$

% and 

% \[
% \begin{cases}
%   (-\Delta + \kappa^2)v_y  &= f_y \text{ in } \Omega \\
%   \beta v_y + \frac{\partial v_y}{\partial n} &= 0 \text{ in } \partial \Omega . 
% \end{cases}
% \]

% Ideally, we'd like to have $u_y \equiv v_y, \ \forall y \in S$. We can't so we take

% $$
% \beta^{*} := \argmin_{\beta > 0} \frac{1}{2} \sum_{y \in S} \int_{\Omega}(u_y(x) - v_y(x) )^2dx.
% $$ 
% Now, take finer and finer discretizations. As $S \to \Omega$ (in the sense that $S$ becomes
% dense in $\Omega$), we get $f_y \to \delta_{y}, \ \sum_{y \in \Omega} \to \int_{\Omega}, \ u_y \to \Phi(\cdot, y)$
% and $v_y \to G(\cdot, y)$. Consequently,
% the optimization problem becomes

% $$
% \beta^{*} = \argmin_{\beta > 0} \frac{1}{2} \int_{\Omega}\int_{\Omega} (\Phi(x,y) - G(x,y) )^2 dx dy,
% $$

% which recovers the previous, more natural optimization problem.

\bibliographystyle{unsrt}
\bibliography{refs.bib}
\end{document}
