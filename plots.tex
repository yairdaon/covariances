\documentclass{article}

\usepackage[T1]{fontenc} % Use 8-bit encoding that has 256 glyphs
\usepackage[english]{babel} % English language/hyphenation
\usepackage{amsmath, amsfonts, amsthm} % Math packages

%\usepackage[sort&compress,square,comma,authoryear]{natbib}

% makes color citations
\usepackage[
  %dvips,dvipdfm,
  colorlinks=true,urlcolor=blue,citecolor=red,linkcolor=red,bookmarks=true]{hyperref}

\usepackage{color}
\usepackage{pgfplots}
\usepackage{tikz}
%\pgfplotsset{compat=1.9} 
%\usepackage{hyperref}
%\usepackage{algorithm} % algorithm package
%\usepackage[noend]{algpseudocode}
\usepackage{graphicx}
%\usepackage{wrapfig}
\usepackage{paralist}
\usepackage{graphics} %% add this and next lines if pictures should be in esp format
\usepackage{epsfig} %For pictures: screened artwork should be set up with an 85 or 100 line screen
\usepackage{graphicx} 
\usepackage{epstopdf} 
\usepackage[colorlinks=true]{hyperref}

%----------------------------------------------------------------------------------------
% new commands
%----------------------------------------------------------------------------------------
\newcommand{\der}{\text{d}}
\newcommand{\coder}[1]{\texttt{#1}}
\newcommand{\inner}[2]{#1 \cdot #2}
\newcommand{\var}{\text{Var}}
\newcommand{\corr}{\text{Corr}}
\newcommand{\cov}{\text{Cov}}
\newcommand{\diag}{\text{diag}}
\newcommand{\dom}{\mathcal{D}om}
\newcommand{\gsnote}[1]{{\textcolor{blue}{#1}}}
\newcommand{\ydnote}[1]{{\textcolor{cyan}{#1}}}

\newcommand{\precop}{\mathcal{L}}
\newcommand{\Op}{\mathcal{A}}
\newcommand{\covop}{\mathcal{C}}
\newcommand{\lag}{\mathcal{L}}
\newcommand{\n}{\boldsymbol{n} }
\newcommand{\dn}{\partial \n }
\newcommand{\x}{{\boldsymbol{x}}}
\newcommand{\y}{{\boldsymbol{y}}}
\newcommand{\z}{{\boldsymbol{z}}}
\newcommand{\kxmy}{ \kappa \| \x - \y \| }
\newcommand{\xmy}{ \| \x - \y \| }
\newcommand{\proj}{\mathcal{P}}
\newcommand{\kr}{\kappa r}

% Choose your citing style here!!!!!!!!!!!!!!!!!!!

\DeclareMathOperator*{\argmin}{arg\,min}

\begin{document}

\begin{figure}
  \minipage{0.52\textwidth}
  \begin{tikzpicture}[thick,scale=.85, every node/.style={scale=0.99}]
    \begin{axis}
      [
      xmin = 0,
      xmax = 0.5,
      xlabel = {$s$},
      ylabel = {$c(\x (s), \x^\star)$},
      ymin   = 0,
      compat = 1.3,
      % ymax   = 130,
      ytick = \empty,
      legend cell align=left
      ]
      \draw[black!30!white, thin] (50,0) -- (50,130);
      % 
      \addplot [thick, black, mark=none] table {data/square/Free_Space_Greens_Function.txt};
      \addlegendentry{\large Free-Space};
      % 
      \addplot[thick, blue, mark=none, dashed] table {data/square/Dirichlet_Greens_Function.txt};
      \addlegendentry{\large Dirichlet BC};
      % 
      \addplot[thick, red, dotted, mark=none] table {data/square/Neumann_Greens_Function.txt};
      \addlegendentry{\large Neumann BC};
      % \node at (0.85cm,0.2cm) {$\y$};
    \end{axis}
    \begin{axis}
      [
      compat=1.3,
      axis lines = none,
      xmin = -0.1,
      xmax = 1.1,
      ymin   = -0.1,
      ymax   = 1.1,
      xtick = \empty,
      ytick = \empty,
      height = 4.5cm,
      width = 4.5cm,
      at={(3.9cm,1.2cm)},
      legend cell align=left
      ]
      \addplot [thick, black!50!white, mark=none, fill=black!10!white] table {data/square/vertices.txt};
      \addplot [thick, red  , mark=none] table {data/square/line.txt};
      \addplot [only marks, mark = *, mark size=1.2] table {data/square/source.txt};
      \node at (2.4cm,0.5cm) {$\Omega$};
      \node at (0.6cm,1.8cm) {$\x^\star$};
      \draw [black, thin] (0.35cm,1.45cm) -- (0.45cm,1.65cm);
      \draw [black, thin] (0.6cm,1.45cm) -- (0.8cm,1.2cm);
      \node at (1.4cm,1.1cm) {cross section};
    \end{axis}
  \end{tikzpicture}
  \endminipage\hfill
  \caption{Left: Cross sections through covariance functions induced
    by elliptic PDE operators with different boundary
    conditions. Shown is also a sketch of the domain $\Omega=[0,1]^2$
    and the cross section $\x(s) = (s, 0.5)^T$.  The center is located
    at $\x^\star = \x(0.05) = (0.05, 0.5)^T$.}
  \label{fig:problem illustration}
\end{figure}



% \begin{figure}
%   \begin{tikzpicture}[thick,scale=0.75, every node/.style={scale=1}]
%     \begin{axis}
%       [
%       compat = 1.3,
%       % xmin = 0.05,
%       % xmax = .95,
%       ymin = 8,
%       xlabel = {$s$},
%       ylabel = {$\beta((0,s)^T)$},
%       legend style={at={(0.5,0.97)},anchor=north},
%       legend cell align=left,
%       ]
%       \addlegendimage{empty legend}
%       \addlegendentry{\hspace{-.6cm}{Square, $L^2$-proj.}}
%       \addplot
%       [mark=triangle, red, only marks, mark size=.7]
%       table {data/square/beta_radial_32.txt};
%       %
%       %
%       \addplot
%       [mark=o, blue, only marks, mark size=.7] table
%       {data/square/beta_radial_64.txt};
%       %
%       %
%       \addplot
%       [mark=square, green, only marks, mark size=.7]
%       table {data/square/beta_radial_128.txt};
%       %
%       %
%       \addplot
%       [very thick, mark=none, black] 
%       table {data/square/beta_adaptive_2.txt};
%       \node at (0, 50) {\Large (a)};
%       \addlegendentry{$n=32$}
%       \addlegendentry{$n=64$}
%       \addlegendentry{$n=128$}
%       \addlegendentry{adaptive}
%     \end{axis}
%   \end{tikzpicture}\hfill
% %
%   \begin{tikzpicture}[thick,scale=0.75, every node/.style={scale=1}]
%     \begin{axis}
%       [
%       compat = 1.3,
%       % xmin = 0.05,
%       % xmax = .95,
%       xlabel = {$s$},
%       ylabel = {$\beta((0,1/2,s)^T)$},
%       legend style={at={(0.5,0.97)},anchor=north},
%       legend cell align=left
%       ]
%       \addlegendimage{empty legend}
%       \addlegendentry{\hspace{-.6cm}{Cube, $L^2$-proj.}}
%       \addplot
%       [red, mark=triangle, only marks, mark size = .7]
%       table {data/cube/beta_radial_32.txt};
%        %
%       %
%       \addplot
%       [blue, mark=o, only marks, mark size=.7] 
%       table {data/cube/beta_radial_64.txt};
%       %
%       %
%       \addplot
%       [green,  mark=square, only marks, mark size=.7]
%       table {data/cube/beta_radial_128.txt};
%       %
%       %
%       \addplot
%       [very thick, black, mark=none]
%       table {data/cube/beta_adaptive_2.txt};
%       \node at (0, 0) {\Large (b)}; 
%       \addlegendentry{$n=32$}
%       \addlegendentry{$n=64$}
%       \addlegendentry{$n=128$}
%       \addlegendentry{adaptive}
%     \end{axis}
%   \end{tikzpicture}\\
%  \begin{tikzpicture}[thick,scale=0.75, every node/.style={scale=1}]
%     \begin{axis}
%       [
%       compat = 1.3,
%       % xmin = 0.05,
%       % xmax = .95,
%       ymin = 8,
%       xlabel = {$s$},
%       ylabel = {$\beta((0,s)^T)$},
%       legend style={at={(0.5,0.97)},anchor=north},
%       legend cell align=left
%       ]
%       \addlegendimage{empty legend},
%       \addlegendentry{\hspace{-.6cm}{Square, direct int.}}
%       \addplot
%       [mark=triangle, red, only marks, mark size=.7]
%       table {data/square/beta_std_32.txt};
%       %
%       %
%       \addplot
%       [mark=o, blue, only marks, mark size=.7] table
%       {data/square/beta_std_64.txt};
%       %
%       %
%       \addplot
%       [mark=square, green, only marks, mark size=.7]
%       table {data/square/beta_std_128.txt};
%       %
%       %
%       \addplot
%       [very thick, mark=none, black] 
%       table {data/square/beta_adaptive_2.txt};
%       \node at (0, 15) {\Large (c)}; 
%       \addlegendentry{$n=32$}
%       \addlegendentry{$n=64$}
%       \addlegendentry{$n=128$}
%       \addlegendentry{adaptive}
%     \end{axis}
%   \end{tikzpicture}\hfill
% %
%   \begin{tikzpicture}[thick,scale=0.75, every node/.style={scale=1}]
%     \begin{axis}
%       [
%       compat = 1.3,
%       % xmin = 0.05,
%       % xmax = .95,
%       xlabel = {$s$},
%       ylabel = {$\beta((0,1/2,s)^T)$},
%       legend cell align=left,
%       legend style={at={(0.5,0.97)},anchor=north},
%       ]
%       \addlegendimage{empty legend},
%       \addlegendentry{\hspace{-.6cm}{Cube, direct int.}}
%       \addplot
%       [red, mark=triangle, only marks, mark size = .7]
%       table {data/cube/beta_std_32.txt};
%       %
%       %
%       \addplot
%       [blue, mark=o, only marks, mark size=.7] 
%       table {data/cube/beta_std_64.txt};
%       %
%       %
%       \addplot
%       [green,  mark=square, only marks, mark size=.7]
%       table {data/cube/beta_std_128.txt};
%       %
%       %
%       \addplot
%       [very thick, black, mark=none]
%       table {data/cube/beta_adaptive_2.txt};
%       \node at (0, 0) {\Large (d)}; 
%       \addlegendentry{$n=32$}
%       \addlegendentry{$n=64$}
%       \addlegendentry{$n=128$}
%       \addlegendentry{adaptive}
%     \end{axis}
%   \end{tikzpicture}
%   \caption{Optimal Robin boundary coefficients $\beta$ for an edge of
%     a square (a), (c) and a line on a face of a cube (b), (d). Shown
%     are coefficients computed by adaptive quadrature, and their
%     discrete approximations on regular meshes obtained by dividing
%     $n^2$ squares into $4n^2$ triangles in two dimensions, and $n^3$
%     cubes into $6n^3$ tetrahedra in three dimensions. The
%     approximations are either based on approximate $L_2$-projections followed
%     by finite element quadrature (a), (b) or on direct finite element
%     quadrature (c), (d) as discussed in the paper.
%     \label{fig:beta}}
% \end{figure}

\begin{figure}
  \begin{tikzpicture}[thick,scale=1, every node/.style={scale=1}]
    \begin{axis}
      [
      % yscale = 0.75,
      % xscale = 0.75,
      compat = 1.3,
      xmin = 0.004,
      xmax = 0.5,
      xlabel = {$s$},
      ylabel = {$c(\x^{\star}, \x(s) )$},
      ymin   = 0,
      ymax   = 0.0008,
      %title = {optimal Robin BC},
      yticklabels = , 
      ytick = \empty,
      legend style={nodes=right},
      legend style={at={(1,1)},anchor=north east},
      legend cell align=left
      ]
      \draw[black!30!white, thin] (25,0) -- (25,130);
      %
      %
      \addplot 
      [thick, black, mark=none]
      table
      {data/parallelogram/Free_Space_Greens_Function.txt};
      \addlegendentry{Free-Space};
      %
      %
      %
      \addplot 
      [thick, red, dotted, mark=none] 
      table 
      {data/parallelogram/Neumann_Greens_Function.txt};
      \addlegendentry{Neumann BC};
      % 
      %
      %
      % \addplot 
      % [thick, green!80!black, loosely dashdotted, mark=none] 
      % table 
      % {data/parallelogram/Roininen_Greens_Function.txt};
      % \addlegendentry{Constant Robin}
      %
      %
      %
      \addplot 
      [thick, green!60!black, densely dashdotted       ,   mark=none] 
      table 
      {data/parallelogram/Ours_Greens_Function_Radial.txt};
      \addlegendentry{Var.\ Robin};
      
      % \addplot 
      % [red, thick, dashed, mark=none] 
      % table 
      % {data/parallelogram/Ours_Constant_Variance_Greens_Function_Radial.txt};
      % \addlegendentry{Var.\ Robin+Const.\ Var.};
      
      % \addplot 
      % [thick, blue, dashed, mark=none] 
      % table 
      % {data/parallelogram/Neumann_Constant_Variance_Greens_Function.txt};
      % \addlegendentry{Neumann+Const.\ Var.};
    \end{axis}
  \end{tikzpicture}
  %%
  %%
   \begin{tikzpicture}[thick, scale = 1, every node/.style={scale=1}]
     \clip (0,-0.85) rectangle (3.2,10);
     \begin{axis}
      [
      % yscale = 0.75,
      % xscale = 0.75,
      xmin = -0.01,
      xmax = .5,
      ymin   = -0.01,
      ymax   = .5,
      % xtick = \empty,
      % ytick = \empty
      xlabel = {$$},
      legend cell align=left
      ]      
      \addplot [mark=none, fill=black!10!white] table {data/parallelogram/vertices.txt}; 
      \addplot [thin, black, mark=none] table {data/parallelogram/vertices.txt};
      \addplot [thick, red  , mark=none] table {data/parallelogram/line.txt};
      \addplot [only marks, mark = *   ] table {data/parallelogram/source.txt};
      \node at (2.4cm,3.cm) {\large $\Omega$};
      %
      \node at (1.1cm,1.1cm) {$\x^\star$};
      \draw [black, thin] (0.9cm,0.9cm) -- (0.42cm,0.35cm);
      %
      \node at (1.8cm,2.1cm) {cross section};
      \draw [black, thin] (1.6cm,1.95cm) -- (1.6cm,0.95cm);
      \draw[black] (23.75,0) -- (23.75,130);
    \end{axis}
  \end{tikzpicture}
  \caption{The left plot shows covariance functions derived from PDE
    operators with different boundary conditions for the
    parallelogram domain example.
    Shown are slices of the Green's funcation along a cross section.
    The right plot shows part of the parallelogram domain $\Omega$.
    The black dot is
    $\x^{\star} = (0.025, 0.025)^T$---the center of the
    Green's functions. 
    The red line indicates the cross section
    $\x(s) = (s, 0.6s + 0.01 )$, which is used in the left plot.
  \label{fig:parallelogram greens}}
\end{figure}

\end{document}

